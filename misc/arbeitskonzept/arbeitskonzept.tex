\documentclass[12pt, a4paper]{article}
\usepackage[T1]{fontenc}

\usepackage[english]{babel}

\usepackage{graphicx}%including graphics/images
\usepackage{amsmath}%for math
\usepackage{amssymb}%also for math
\usepackage{fancyhdr}%for header and footer
\usepackage{xcolor}%for color settings
\usepackage[a4paper, portrait, margin=2cm]{geometry}%for page layout
%\usepackage[compact]{titlesec}%for sections
\usepackage[normalem]{ulem}%for strikethrough text

%footer and header
\pagestyle{fancy}
\fancyhf{}
\renewcommand{\headrulewidth}{2pt}
\rhead{Luis Wirth}
\lhead{Arbeitskonzept}
\chead{Maturarbeit}
\cfoot{\thepage}

\begin{document}
\sloppy

\section*{Arbeitskonzept von Luis Wirth}

\subsection*{Titel der Arbeit}
``Bildsynthese von menschlichen Gesichtern mit KI''

\subsection*{Leitfrage}
``Wie generiert man mithilfe von Machine Learning künstliche Bilder von menschlichen Gesichtern?''

\subsection*{Gegenstand der Untersuchung}
\textbf{Machine Learning} ist momentan ein sehr aktueller Themenbereich der viele neue Möglichkeiten bietet.
Es vereint die beiden Gebiete der Informatik und der Mathematik, um komplexe Problemstellungen zu lösen.
Die vorliegende Arbeit soll beispielhaft eine konkrete \textbf{Anwendung} von Machine Learning aufzeigen, namentlich die künstliche Bildgenerierung von menschlichen Gesichtern.
\\\\
Zu diesem Zweck gelangen \textbf{künstliche Neuronale Netze} zum Einsatz. Dabei wird der Fokus auf einer spezifischen Architektur liegen: den \textbf{Autoencodern}.
Um das Vorgehen zu verstehen, wird die relevante Theorie erläutert.

\subsection*{Fachliche Verfahren}
Für das Programmieren der Applikation, welche die Bildgenerierung ausführen soll, wird \textbf{Python3} verwendet.
Mithilfe von \textbf{Tensorflow}, ein von Google entwickeltes Framework für datenstromorientierte Programmierung, wird das Neuronale Netz implementiert.
\textbf{Keras}, eine Schnittstelle von Tensorflow, wird das Erstellen des Computational Graphs des Neuronalen Netzes erleichtern. 
\\\\
Dank Tensorflow und Keras müssen die komplexen Verfahren zum Trainieren von Neuronalen Netzen (\textbf{Gradient Descent} in Kombination mit  \textbf{Linearer Algebra}) nicht selbst implementiert werden.
Jedoch wird die Maturarbeit die dazugehörige Theorie darlegen.


\subsection*{Ressourcen}
Für das Trainieren des Neuronalen Netzes ist ein leistungsstarker Computer (d.h. mit guter GPU) von Vorteil.
\\\\
Des weiteren werden \textbf{Trainingsdaten} für das Trainieren des Neuronale Netz benötigt. In diesem Fall handelt es sich um Fotos von menschlichen Gesichtern, welche alle ein ähnliches Format und einen möglichst uniformen Hintergrund haben sollten.
Dies könnten zum Beispiel Jahrbuchbilder einer Schule sein oder herausgefilterte Bilder aus dem Internet.
\\\\
Informationsquelle wird vor allem das Internet sein, da dieses Gebiet schwerpunktmässig dort behandelt wird. Für die notwendige Theorie zu Gradient Descent und dergleichen werden passende Bücher konsultiert.

\pagebreak

\subsection*{Zielsetzung}
Das Ziel der Arbeit ist es, die Grundlagen und Möglichkeiten von Machine Learning anhand eines konkreten Anwendungsbeispiel darzulegen.
Dabei soll ebenfalls aufgezeigt werden, welcher Aufwand mit einem solchen Projekt verbunden ist.
Die Arbeit kann deshalb als Orientierungshilfe für ähnliche Projekte dienen.\\\\
Gleichzeitig soll verdeutlicht werden, dass es auch für programmieraffine Laien grundsätzlich möglich ist, Machine Learning auf komplexe Problemstellungen anzuwenden.
\\\\
Durch die Identifikation von verbindenen Merkmalen eines Datensatzes, ist es möglich, \textit{neue} Daten mit den gleichen Merkmalen zu generieren.
Dieses Prinzip hat unzählige Anwendungsmöglichkeiten! Einige davon werden in der Arbeit erläutert werden.
Autoencoder weisen zudem weitere interessante Aspekte auf, von denen einige in der Arbeit behandelt werden.

\subsection*{Zeitplan}
Deadlines
\\\\
\begin{tabular}{r | l}
    29.03.19 & Abgabe Arbeitskonzept\\
    12.04.19 & Theorie zu Machine Learning schreiben\\
    28.04.19 & Trainingsdaten beschaffen\\
    15.05.19 & Autoencoder programmieren\\
    20.05.19 & Face-Aligner programmieren\\
    25.05.19 & Trainieren und Hyperparameter einstellen\\
    15.06.19 & Anpassungen am Model?\\
    15.07.19 & Autoencoder ausbauen? $\rightarrow$ VAE\\
    01.09.19 & fertig schreiben\\
    20.09.19 & Abbildungen in Ti\textit kZ machen\\
    26.09.19 & Korrektur\\
    27.09.19 & Abgabe Maturarbeit\\
    10.12.19 & Präsentation vorbereiten\\
    16.12.19 & MA-Präsentation\\

\end{tabular}\\

\end{document}
